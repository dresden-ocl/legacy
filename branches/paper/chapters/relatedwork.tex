\section{Related Work}
\label{sec:relatedWork}
In \cite{kolovos2006queries} a first approach to reuse OCL semantics at the model instance level
for various model realisations was proposed. The approach presented in \cite{kolovos2006queries}
allowed to reflect properties on model instances and thus, to query on them. In contrast, the
approach presented in this paper supports the invocation
of model-defined operations on instances as well. 

\note{Claas: We have to adapt the following sentences.}
In the following we will discuss alternative tools to parse,
interpret, or compile OCL constraints and the means they provide
to support variability in models and model instances:

\begin{itemize}
  
\item The \emph{USE} tool \cite{gogolla2007use} contributes an OCL simulator that can
evaluate OCL constraints against model snapshots. It is bound to UML class, UML object and
UML sequence diagrams and does not provide means for model or model instance
adaptations. Nevertheless, a case study proofed that it is possible to create snapshots
from Java runtime objects that can be evaluated with USE \cite{occello:ICSTW08}.

\item The \emph{OCLE} tool~\cite{WWW:OCLE} interprets OCL constraints on
UML models. Furthermore, it provides a compiler to generate a Java
implementation from a constrained UML model and the according OCL 
constraints. Model adaptation is not supported. Although OCLE does not allow for
real model instance adaptation, XML files can be treated as model instances by
transforming them into UML object diagrams.

\item The \emph{MIP OCL2 Parser}~\cite{WWW:MIP} is a Java library for parsing OCL
constraints provided by the Institute for Defense Analyses. Constraints are
checked syntactically and semantically against a UML class diagram.
To use the parser, one must provide a Java implementation of the abstract 
UML model expected by the parser. Thus,
the MIP parser provides very limited means for model adaptation. Since MIP does
not contribute an interpreter or compiler for constraints, model
instance adaptation is not relevant.

\item The OCL interpreter and compiler provided by the \emph{Kent Modeling Framework
(KMF)} supports model adaptation via a central \emph{Bridge}
model~\cite{akehurst2003ocl}. Both, the compiler and the interpreter depend on a
Java-based representation of model instances. Thus, model instance adaptation is not
supported.

\item The \emph{Epsilon Validation Language (EVL)} introduced in
\cite{kolovos2008detecting} is quite similar to OCL. It comes with an
interpreter that can be used for various EMF-based languages. Thus, model
adaptation is possible while model instances are bound to EMF.
 
\item A standard OCL interpreter for EMF is provided by \emph{MDT OCL}
\cite{WWW:MDT}. It is also tightly integrated with EMF and supports 
model adaptation for various EMF languages. The interpreter directly supports
model instances represented in EMF. MDT OCL's architecture is highly extensible
and could be adapted to other model instances using \emph{Java Generics} \cite{damus:EclipseCon08}. 
However, we are not aware of any such adaptations.

\end{itemize}

This analysis of related work shows that variability at \add{the} model level is
considered useful and has already been implemented in various OCL tools.
Supporting variability at the model instance level -- as suggested in this paper -- 
is a consequent continuation of our previous and other's related work.