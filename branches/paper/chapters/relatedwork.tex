\section{Related Work}
In the following we will discuss alternative tools to parse,
interpret, or compile OCL constraints and the means they provide
to support variability in models and model instances:

\begin{itemize}
  
\item The USE tool \cite{gogolla2007use} contributes an OCL simulator that can
evaluate OCL constraints against model snapshots. It is bound to UML class, UML object and
UML sequence diagrams and does not provide means for model or model instance
adaptations.

\item The OCLE tool~\cite{WWW:OCLE} interprets OCL constraints on
UML models. Furthermore, it provides a compiler to generate a Java
implementation from a constrained UML model and the according OCL 
constraints. OCLE does not allow for model or model instance adaptation.

\item The MIP OCL2 Parser~\cite{WWW:MIP} is a Java library for parsing OCL
constraints provided by the Institute for Defense Analyses. Constraints are
checked syntactically and semantically against an UML class diagram.
To use the parser, one must provide an Java implementation of the abstract 
UML model expected by the parser. Thus,
the MIP parser provides very limited means for model adaptation. Since MIP does
not contribute a interpreter or compiler for constraints, model
instance adaptation is not relevant.

\item The OCL interpreter and compiler provided by the Kent Modeling Framework
(KMF) supports model adaptation via a central \emph{Bridge}
model~\cite{akehurst2003ocl}. Both depend on a Java-based
representation of model instances. Thus, model instance adaptation is not
directly supported.

\item The Epsilon Validation Language (EVL) introduced in
\cite{kolovos2008detecting} is quite similar to OCL. It comes with an
interpreter that can be used for various EMF-based languages. Thus, model
adaptation is possible while model instances are bound to EMF.
 
\item A standard OCL interpreter for EMF is provided by MDT OCL
\cite{WWW:MDT}. It is also tightly integrated with EMF and supports 
model adaptation for various EMF languages. The interpreter directly supports
model instances represented in EMF. Its architecture is highly extensible
\note{add cite} and could be adapted to other implementation
infrastructures. However, we are not aware of any such adaptations.

\end{itemize}

This analysis of related work shows that variability at model level is
considered useful and has already been implemented in various OCL tools.
Supporting variability at the model instance level -- as suggested in this paper -- 
is, thus, a consequent continuation of our previous
and other's related work.



% MDT OCL, USE, BluePrint OCL, AspectJ, ...
% 
% In \cite{hartmannDEXA06,hartmannICWI07} Hartmann presented an approach for XML-based template engines that showed a use case for OCL constraints on XML Schemas. OCL was used to express constraints that cannot be defined in XML Schema.
% 
% OCL on XML/DTD:
% \begin{itemize}
% 	\item http://lci.cs.ubbcluj.ro/ocle/
% 	\item Modeling XML applications with UML (D. Carlson) (Buch)
% \end{itemize}
% 
% \textbf{TODO}