\section{Challenges}

\subsection{Limitations of Implementation Adaptation}
	\note{Could we move this section to a later section. It is better to first
	show how cool the approach already works and then reflect on its limitations. 
	Furthermore, knowing concrete examples may help understanding the limitations}
	\note{The whole section is hard to comprehend. I would suggest to use named
	paragraphs for each concrete limitation} During the implementation of our
	generic architecture that allows connection of different types of models 
	with different types of model implementations and \note{vice versa
	direction was not introduced before}{vice versa} we met various problems and
	difficulties that are presented in this section. 

	\paragraph{Problem Name}
	As mentioned above, models and model implementations are only loosely 
	coupled to support different implementations for the same type of model.
	\note{don't understand argument: loose coupling ENABLES variable
	implementations} Thus, we have to match types of an concrete implementation to
	the types of a model. Currently, this type matching is realized by simply 
	matching the names of the types (including the names of 
	their enclosing name spaces where possible). Often, this solution is error 
	prone and further information is hidden behind our adapters that could be 
	used to improve the matching. E.g., when adapting an instance of an EMF 
	Ecore model, one could use the reflection mechanisms provided by Ecore to
	retrieve the types of EObjects. We plan to improve our matching
	that is currently scattered over multiple classes of each implementation type 
	adapter and by introducing type matching strategies that 
	can be implemented using the \emph{Chain of Responsibility} 
	pattern\cite{gamma:dp}. The chain could start by trying to match the 
	types using a implementation specific matcher and end by trying to 
	simply match the type's names as currently done.
	
	\paragraph{Problem Name}
	Another problem when using adapters for model implementation types 
	is the unwrapping mechanism of adapted elements when invoking operations 
	on the implementation's elements. E.g., to invoke an operation of a Java 
	implementation we require \texttt{Objects} as parameters instead of 
	\texttt{IModelInstanceElements}. This unwrapping mechanism is easy 
	where elements that have been adapted before are simply unwrapped 
	again. Unfortunately, during interpretation of OCL constraints, 
	new instances of primitive types or new collections can be created 
	(e.g., when invoking the OCL operation \texttt{size()} that 
	returns an \texttt{Integer} instance). Thus, the factory of a 
	model implementation adapter has to provide operations to reconvert 
	primitive types and collections to elements of the adapted model 
	instances. In some cases this can become rather complicate because 
	the adaptation between types of the implementation and the model 
	implementation type interfaces has not to be bijective. For 
	example Java \texttt{ints} and \texttt{java.lang.Integers} are 
	both mapped to \texttt{IModelInstaceIntegers}. During unwrapping, 
	the factory of the Java implementation adapter has to reflect 
	whether the method to invoke requires an \texttt{int}, an 
	\texttt{Integer} or another Java integer-like type's instance.
	 The unwrapping mechanism of an adapted implementation 
	 can be considered as the most complicate and error-prone part 
	 of the complete implementation type adaptation. \remove{Nevertheless, 
	 this part is necessary to support interpretation of OCL 
	 constraints containing operation invocations on model defined 
	 types.}\note{What are we planning to alleviate this challenge?}


	\subsection{Improving the Adaptation Process}
	\note{can we merge this with the limitations section}
	
	\paragraph{Problem Name}
	Adapting different types of meta-models\add{, models, } and model
	implementations enables the reuse of DrescenOCL for \add{various combinations
	of modeling and implementation languages}.
	Nevertheless, the adaptation process contains parts that are similar 
	for each adaptation and can be error
	prone as well. \note{how are similar and error-prone connected?}
	\note{rephrase}
	When the adaptation of a model implementation 
	type is wrong, the interpretation results may be wrong as well. 
	To improve the model adaptation process, we developed a code generator 
	for the adaptation of meta-models to the pivot model. The code 
	generator requires an annotated \change{EMF Ecore model}{metamodel} describing
	the adaptation of meta-model concepts to the pivot model concepts. 
	The code generator generates the skeleton code for all required 
	adapters and thus avoids manual implementation of these adapters. 
	For model implementation types, such a code generator is currently 
	missing.
	
	\paragraph{Problem Name}
	\remove{Furthermore,} We developed two generic JUnit test suites, 
	that can be used to test the \remove{correct} adaptation of a model 
	or model implementation, respectively. \note{Next sentence is hard to
	comprehend\ldots}The test suites require a specific test model defined in the
	adapted meta-model or a specific test implementation implemented in the 
	adapted model implementation. 
	The test suites then check if all required methods to 
	retrieve on types, operations, properties etc. are implemented 
	appropriately in respect to the specified test model or test 
	implementation. These generic test suite helped us to ensure 
	that all existing adaptations behave in the same expected manner
	and to easily detect wrong adaptations of some elements.
	\note{any future plans?}