\section{Introduction}

Model-Driven Software Development (MDSD) aims to abstract from concrete 
software implementations and uses modeling languages to describe 
software systems. Modeling languages include the Unified Modeling Language 
(UML) \cite{spec:UML2-2}, \note{Ecore is a metamodeling language}{Ecore} of the
Eclipse Modeling Framework (EMF) \cite{WWW:EMF} and domain-specific modeling languages (DSLs). The Object 
Constraint Language (OCL) \cite{spec:OCL2-2} was originally designed as 
an extension of the UML to express structural and functional constraints 
on UML models in a textual syntax. In recent years, OCL has become a 
popular constraint language for other modeling languages like EMF Ecore 
and various DSLs as well\note{need references here}. 

OCL can be
defined and used as a constraint language \change{on top of}{in combination
with} every modeling language, that \change{contains  basic}{involves
object-oriented} modeling concepts like types and navigable properties.
When integrating OCL with such a modeling language, the concrete and 
abstract syntax of OCL is not \change{altered}{affected}. As OCL is statically
typed, the abstract syntax only needs to access to the types of the model 
it is defined on. \add{Thus, a generic implementation of an OCL parser
that relies on an abstraction of the underlying model type system can be
reused to parse OCL constraints on different types of
models}\cite{braeuerOCL07}.
%Such a generic architecture for an OCL tool was presented by Matthias Br\"auer 
%in 2007 \cite{braeuerOCL07}. %Besides the abstract syntax and static semantics, 
%the OCL specification defines the dynamic OCL semantics as well
%\cite{spec:OCL2-2}.

One approach to evaluate 
%the dynamic semantics of 
OCL constraints on instances 
of constrained models is \textit{Constraint Interpretation}. \note{Incorporate
remark that discusses the benefit of interpreters compared with compilers
w.r.t. language flexibility}
\note{Introduce the notion model implementation|instantiation. Motivate why it
is good to vary implementation, reference examples}
% Although OCL is 
% defined abstractly and independent of specific implementations, 
% and can be used at different meta levels,\footnote{E.g., OCL can be used to 
% define \textit{Well-Formedness Rules (WFRs)} on meta-models checked on models.
% Alternatively, OCL can be used to define business rules on models that are
% checked on the model's instances.}
OCL interpreters are \change{normally limited}{typically bound} to a specific
\change{programming or modeling}{implementation} languages. Thus, a
reimplementation of OCL's dynamic semantics is required for every type of 
model \change{implementation}{instantiation}. 
\remove{Each reimplementation leads to
code duplication, limits reuse and makes it more difficult to maintain or extend 
the OCL semantics for all types of model instances synchronously.}

In this paper, we propose an architecture for an OCL
\change{tool}{infrastructure} that \change{encapsulates}hides meta-models,
models and model instantiations behind well-defined interfaces. \change{Thus, it is possible to}{This enables} reuse
\add{of} the complete OCL infrastructure including the parser, standard library
and interpreter for \change{every combination of different model and
instantiation types}{various model implementation languages}. \change{An
implementation of the approach is used in our tool}{We present an
implementation of our approach for} \textit{DresdenOCL}
\cite{WWW:dresdenOCL}\remove{, released in its third version under the name of
\textit{Dresden OCL2 for Eclipse}}. \textit{DresdenOCL} is able to parse OCL
constraints on different types of models such as UML, EMF Ecore, Java and XML Schema 
and evalute them for different model implementations representing model
instances as Java objects, XML files or \note{confusing\ldots}{UML diagrams}.

\textbf{TODO The paper is structured as follows...}