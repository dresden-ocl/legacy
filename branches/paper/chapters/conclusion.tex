\section{Conclusion}
\label{sec:conclusion}

In this paper we presented a \add{generic} approach for \remove{an} OCL
\change{interpreter}{interpretation} that supports both model and
model instance variability. Various OCL infrastructures support 
model variability, whereas -- to the best of our knowledge -- none of 
the existing OCL infrastructure supports \remove{full} model instance
variability. Our approach \change{solves this lack}{addresses this problem} by
abstracting from domain-specific concepts and by introducing well-defined interfaces for models
their instances. \change{By designing}{With our implementation of} such a
generic adaptation architecture, the same OCL interpreter was applied to
\add{three} case studies that are located at different modelling layers and
use different \change{model realisations}{models and model instances}. Thus, we
\change{improved the reuse of the} contribute a reusable OCL interpreter and
avoided new implementations of the OCL standard library for various different
\change{model instances}{technical spaces}. \remove{Three case studies proofed
the feasibility of our approach.}

\change{Nevertheless, some task remain open for future work. In fact, }{For
future work} we plan to improve our approach by addressing the gaps mentioned 
in Sect. \ref{sec:lessons}. \add{Furthermore, we are interested in evaluating
the performance impact of our adapter-based approach for OCL interpretation.
Therefore we plan a benchmark comparing our interpreter with other
OCL interpreters and compilers and a continuation of our previous}
work~\cite{OCLRelDB} on extensible OCL compilation.


\section{Acknowledgements}

We want to thank Tricia Balfe of \textsc{Nomos Software} for providing data for the XML case study and for continuous feedback during adaptation of the case study.
Furthermore, we'd like to thank all people that are or were involved into the DresdenOCL project.