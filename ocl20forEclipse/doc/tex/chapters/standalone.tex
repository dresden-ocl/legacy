\chapter{Standalone -- Using Dresden OCL outside of Eclipse}
\label{chapter:standalone}
\lstset{language=Java}

\begin{flushright}
\textit{Chapter written by Michael Thiele}
\end{flushright}

Although Dresden OCL is targeted at the Eclipse platform, it can be used outside
of it as a standalone application. All GUI components cannot be used, but all 
the core components like model loading, model instance loading, parsing OCL 
constraints, interpreting them or generating AspectJ code out of them are 
available.


\section{The Example Application}

The easiest way to use the standalone application is to check out the provided 
example at 
\code{http://svn-st.inf.tu-dresden.de/svn/dresdenocl/trunk/ocl20forEclipse/stand\-a\-lone/\newline
tudresden.ocl20.pivot.standalone.example}. It can be used as an Eclipse Java project (not plug-in project), can be imported
into Netbeans or can be used from the command line as an Eclipse runtime is not 
required. The structure of the example is described below.


\subsection{Classpath}

The \reference{lib} folder contains all libraries that are needed when executing
the example. This includes several Eclipse jar files as well as
stringtemplate.jar that is needed for the code generation facilities of the 
application and others. The \reference{plugins} folder contains all jar files
that are provided by Dresden OCL. When using the example as an Eclipse or
Netbeans project all jars are already on the classpath. When using the command
line or other tools, you probably have to add these jars to the classpath manually.


\subsection{Resources}

The \reference{resources} folder has three sub-folders. In
\reference{constraints} the OCL constraint files for three different examples
are located. The \reference{model} folder contains the model files. Note, that 
in order to use the Java models, you need the appropriate \code{.class} files
that are contained in a folder structure that represents the package name of the
classes. Also, all classes that are referenced by the model need to be located
in the folder structure or have to be on the classpath. Otherwise, a 
\lstinline[breaklines=true]{NoClassDefFoundError} will be thrown. The 
\reference{modelInstance} folder contains model instance files that again, in
the case of Java model instances, have the same conditions as described for 
Java models.


\subsection{Logging}

In order to log different parts of Dresden OCL, you can alter the
\reference{log4j.properties} file. There, two appenders and a root logger are 
already defined. In order to log only specific parts of the toolkit, for
example, you can enter: 
\code{log4j.logger.tudresden.ocl20.pivot.in\-ter\-pre\-ter\linebreak[0]=\linebreak[0]debug,
stdout} that puts all log messages of the interpreter on the console (please 
be aware that this will significantly slow down the execution time).


\subsection{Using the Example}

The \reference{src} folder contains one class, named 
\code{StandaloneDresden OCLExample.java}. It contains a \code{main} method that 
executes numerous examples.

\paragraph{Royal \& Loyal}
The first example is the Royal \& Loyal example originally introduced by
{\scshape Warmer \& Kleppe}~\cite{warmerEA:ocl03}. The model is loaded by
calling \code{load\-UML\-Mo\-del\linebreak[0](File, File)} on the
\code{Stand\-a\-lone\-Fa\-cade}. The method takes two arguments, the first being
the file where the UML model is located. The second argument needs to be a file that
points to the resources jar of the Eclipse UML project. In the example project, 
this jar file is located at 
\reference{lib/org.eclipse.uml2.uml.\linebreak[0]re\-sour\-ces\linebreak[0]\_3.0.0.v200906011111.jar}.
The example UML model provided here was built using Eclipse UML 3.x. In order to
load older models, you have to update the namespace in line 2 of the model 
(\code{xmlns:uml="http://www.eclipse.org/uml2/3.0.0/UML"}) or use an older 
resources jar file. You have to provide the resources to be able to use
primitive types in your model that are not mapped correctly when no or the wrong
resources are specified.

Then, the Java model instance is loaded. Note, that this needs to be a class 
that has a method with this signature: \code{public static List<Object> 
getModelObjects()}. In the returned list all objects that shall be part of the 
model instance have to be contained. The actual loading is done by calling 
\lstinline[breaklines=true]{StandaloneFacade.INSTANCE.loadJavaModelInstance(IModel, File)}.

Parsing and interpretation should be self explanatory. The method 
\lstinline[breaklines=true]{StandaloneFacade.INSTANCE.interpretEverything(IModelInstance, List<Constraint>)} returns a list of
\code{I\-In\-ter\-pre\-ta\-tion\-Re\-sult}s that contain the context (\code{getModelObject()}), the executed constraint 
(\code{get\-Con\-straint()}) and the result of the interpretation 
(\code{getResult()}).

In order to generate AspectJ code, one needs to define the
\code{IOcl22CodeSettings} first. The settings can be created by 
\code{Ocl22JavaFactory.getInstance().createJavaCodeGenera\-tor\-Set\-tings()}. 
There are lots of non obligatory settings; only the directory where the code 
should be generated to has to be specified invariably. Afterwards the code can 
be generated calling 
\code{StandaloneFacade.INSTANCE.generateAspectJCode(List<Constraint>, 
IOcl22\-Code\-Set\-tings)}. 

\textbf{Note:} In some projects it is necessary not to generate AspectJ but
plain Java code instead. As Dresden OCL generates plain Java code for the
expressions within OCL constraints---and only their context (e.g., whether the
expression belongs to a definition, an invariant or a precondition) is
transformed into AspectJ code---the plain Java code can be obtained as well. In
this case, the developer is responsible for himself how to integrate the OCL
expressions into his source code. To generate plain Java code instead of
AspectJ code, use the method
\code{generateJavaCode(List\linebreak[0]<Constraint> constraints,
IOcl2JavaSettings settings))}. In this case the method returns a list of Strings
representing the Java code for the individual OCL constraints and does not save
the constraints as Java files within the file system.

\paragraph{PML}
The PML example is nearly analogous to the Royal \& Loyal example described 
above. Before being able to load an Ecore model instance, you have to register 
the Ecore model at the global Ecore registry. In case of PML this is done by 
the lines
\lstset{language=Java}
\begin{lstlisting}
Resource.Factory.Registry.INSTANCE.getExtensionToFactoryMap()
  .put("pml", new XMIResourceFactoryImpl());
PmlPackage.eINSTANCE.eClass();
\end{lstlisting}
Other Ecore models require other factory mappings and their package to be 
initialised.

\paragraph{Simple}
The difference to the other examples is the model instance that is not  loaded
from a class file, but built in the example itself. To create an empty Java 
model instance call
\lstset{language=Java}
\begin{lstlisting}
IModelInstance modelInstance = new JavaModelInstance(model);
\end{lstlisting}
Calling the method \lstinline[breaklines=true]{addModelInstanceElement(Object)} 
on the model instance will cause the given object to be adapted to the model 
instance. Thus, constraints can be evaluated on it.



\section{The Standalone Facade}

The example already contains a jar file of the the facade. In order to change 
the implementation of the \code{StandaloneFacade}, check out the Java project 
available in the SVN 
\code{https://dresden-ocl.svn\linebreak[0].source\-forge\linebreak[0].net/\linebreak[0]svnroot/dresden-ocl/} at \code{trunk/ocl20forEclipse/standalone/tudresden\linebreak[0].ocl\-20\linebreak[0].pi\-vot\linebreak[0].stand\-a\-lone\linebreak[0].facade}.


\subsection{Classpath and OCL Standard Library}
The project contains all required jars in the folders \reference{lib} and 
\reference{plugins} like the example. The \reference{resources} folder also 
needs to be on the classpath. It contains the modelled version of the OCL 
standard library. If you want to use your own version of the OCL standard 
library, you can manipulate the given file or create a new one and change the
\lstinline[breaklines=true]{initialize(URL)} method in the 
\code{StandaloneFacade} to point to your version.


\subsection{Adding and Removing Methods}
The central class for standalone applications is the \code{StandaloneFacade}. 
Adding new methods to the facade should pose no problems as long as all needed
libraries are on the classpath. If the facade offers too much possibilities, 
you can delete all methods you do not need. Then, you are able to remove 
libraries from the classpath that are no longer needed (e.g., all UML related 
jar files when loading UML models is not an option). Thus, you can significantly
reduce the size of needed libraries for standalone applications.


\section{Summary}

The chapter showed how to use Dresden OCL without Eclipse. An extensive example
has been explained and how to modify the standard behaviour of the standalone 
facade that should be used by standalone applications.
