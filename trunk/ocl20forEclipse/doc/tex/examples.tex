% At least, use document class book
\documentclass[a4paper]{article}

\usepackage[T1]{fontenc}
\usepackage[latin1]{inputenc}
\usepackage{listings}
\usepackage{graphicx}
\usepackage{eso-pic}
\usepackage{pstricks}
\usepackage{listings}
\usepackage{graphicx}

\usepackage{color}
\definecolor{blue}{rgb}{0,0,1}

\newcommand{\url}[1]{{\begin{ttfamily}#1\end{ttfamily}}}

\begin{document}

\title{Examples provided with Dresden OCL2 for Eclipse}
\author{Claas Wilke}

\maketitle




This document provides an overview over examples which are provided with \textit{Dresden OCL2 for Eclipse}. How to load models and model instances and how to parse and interpret constraints is explained in a general \textit{Dresden OCL2 for Eclipse Tutorial} which can be found at \textcolor{blue}{http://dresden-ocl.svn.sourceforge.net/\linebreak[0]viewvc/dresden-ocl/trunk/ocl20forEclipse/doc/pdf/tutorial.pdf}.


\begin{figure}[!htbp]
\begin{tabular}[h]{|p{3.2cm}|p{8cm}|}
  \hline
  \textbf{Simple example} & \\
  Plug-in package & \url{tudresden.ocl20.pivot.examples.\linebreak[0]simple}\\
  Meta model & UML 1.5 or UML 2.0\\
  Model & \url{model/simple.xmi} (with UML 1.5) \newline \url{model/simple.uml} (with UML 2.0)\\
  OCL expressions & \url{constraints/invariants.ocl}\\
  Model instance & \url{src/tudresden.ocl20.pivot.examples.simple/\linebreak[0]ModelProviderClass.java}\\
  \hline
  \hline
  \textbf{Living example} & \\
  Plug-in package & \url{tudresden.ocl20.pivot.examples.\linebreak[0]living}\\
  Meta model & UML 1.5\\
  Model & \url{model/UmlExample.xmi}\\
  OCL expressions & \url{expressions/living.ocl}\\
  Model instance & \url{src/tudresden.ocl20.pivot.examples.living/\linebreak[0]ModelProviderClass.java}\\
  \hline
  \hline
  \textbf{PML example} & \\
  Plug-in package & \url{tudresden.ocl20.pivot.examples.\linebreak[0]pml}\\
  Meta model & Ecore\\
  Model & \url{model/pml.ecore}\\
  OCL expressions & \url{expressions/testpml.ocl}\\
  Model instance & \url{model instance/Testmodell.pml}\\
  \hline
\end{tabular}
\end{figure}




\end{document}