\cleardoublepage
\phantomsection
\addcontentsline{toc}{chapter}{Tables}
\chapter*{Tables}

\begin{table}[h]
\begin{tabular}{|p{7cm}|p{7cm}|}
    \hline
    \textbf{Software} & \textbf{Available at} \\
    \hline
    Eclipse 3.5.x & \url{http://www.eclipse.org/} \\
    \hline
    Eclipse Modeling Framework (EMF) & \url{http://www.eclipse.org/modeling/emf/} \\
    \hline
    Eclipse Model Development Tools (MDT) (only with the UML2.0 meta model) & \url{http://www.eclipse.org/modeling/mdt/} \\
    \hline
    Eclipse Plug-in Development Environment (only to run the toolkit using the source code distribution) & \url{http://www.eclipse.org/pde/} \\
    \hline
    AspectJ Development Tools (AJDT) (only to run the generated code of Ocl22Java) & \url{http://www.eclipse.org/ajdt/} \\
    \hline
\end{tabular}
\caption{Software required to run DresdenOCL (v. 2.2.0) with Eclipse \newline(Most parts are con\-tained in the Eclipse MDT Distribution).}
\label{tab:software}
\end{table}



\begin{table}[h]
\begin{tabular}{|p{4cm}|p{10cm}|}
    \hline
    \textbf{Feature} & \textbf{Plug-ins} \\
    \hline

    \textbf{Core} & 
    \textbf{Required:}\newline
    tudresden.ocl20.pivot.logging\newline
    tudresden.ocl20.pivot.essentialocl\newline
    tudresden.ocl20.pivot.essentialcol.edit\newline
    tudresden.ocl20.pivot.essentialocl.editor\newline
    tudresden.ocl20.pivot.essentialocl.standardlibrary\newline
    tudresden.ocl20.pivot.examples.royalandloyal\newline
    tudresden.ocl20.pivot.model\newline
    tudresden.ocl20.pivot.modelbus\newline
    tudresden.ocl20.pivot.modelbus.ui\newline
    tudresden.ocl20.pivot.modelinstance\newline
    tudresden.ocl20.pivot.modelinstancetype\newline
    tudresden.ocl20.pivot.pivotmodel\newline
    tudresden.ocl20.pivot.pivotmodel.edit\newline
    tudresden.ocl20.pivot.standardlibrary.java\newline\newline
    \textbf{Optional:}\newline
    tudresden.ocl20.pivot.essentialocl.tests\newline
    tudresden.ocl20.pivot.modelbus.tests\newline
    tudresden.ocl20.pivot.pivotmodel.tests\newline
    tudresden.ocl20.pivot.standardlibrary.java.tests\newline
    tudresden.ocl20.pivot.testsuite\\
    \hline

    \textbf{Examples} &
    \textbf{Optional:}\newline
    tudresden.ocl20.pivot.examples.living\newline
    tudresden.ocl20.pivot.examples.pain\newline
    tudresden.ocl20.pivot.examples.pml\newline
    tudresden.ocl20.pivot.examples.simple\\
    \hline

    \textbf{Integration Facade} &
    \textbf{Optional:}\newline
    tudresden.ocl20.pivot.facade\\
    \hline

    \textbf{Interpreter} &
    \textbf{Required (for interpretation):}\newline
    tudresden.ocl20.interpreter\newline
    tudresden.ocl20.interpreter.ui\newline\newline
    \textbf{Optional:}\newline
    tudresden.ocl20.interpreter.test\\
    \hline
 
    \textbf{Metamodels} &
    \textbf{Required (at least one of the following):}\newline
    tudresden.ocl20.pivot.metamodels.ecore\newline
    tudresden.ocl20.pivot.metamodels.java\newline
    tudresden.ocl20.pivot.metamodels.uml2\newline\newline
    tudresden.ocl20.pivot.metamodels.xsd\newline\newline
    \textbf{Optional:}\newline
    tudresden.ocl20.pivot.codegen.adapter\newline
    tudresden.ocl20.pivot.metamodels.test\newline
    tudresden.ocl20.pivot.metamodels.ecore.test\newline
    tudresden.ocl20.pivot.metamodels.java.test\newline
    tudresden.ocl20.pivot.metamodels.uml2.test\\
    \hline
 
\end{tabular}
\caption{The plug-ins of DresdenOCL (v. 2.2.0) related to their features (part 1).}
\label{tab:plugins}
\end{table}

\begin{table}[h]
\begin{tabular}{|p{4cm}|p{10cm}|}
    \hline
    \textbf{Feature} & \textbf{Plug-ins} \\
    \hline

    \textbf{Model Instances} &
    \textbf{Required (at least one of the following for interpretation):}\newline
    tudresden.ocl20.pivot.modelinstancetype.ecore\newline
    tudresden.ocl20.pivot.modelinstancetype.java\newline\newline
    tudresden.ocl20.pivot.modelinstancetype.xml\newline
    \textbf{Optional:}\newline
    tudresden.ocl20.pivot.modelinstancetype.test\newline
    tudresden.ocl20.pivot.modelinstancetype.test.ecore\newline
    tudresden.ocl20.pivot.modelinstancetype.test.java\newline
    tudresden.ocl20.pivot.modelinstancetype.test.xml\\
    \hline
 
    \textbf{Ocl2Java} &
    \textbf{Required (for code generation):}\newline
    tudresden.ocl20.pivot.ocl2java\newline
    tudresden.ocl20.pivot.ocl2java.ui\newline
    tudresden.ocl20.pivot.tools.template\newline\newline
    \textbf{Optional (required for code execution):}\newline
    tudresden.ocl20.pivot.ocl2java.types\newline\newline
    \textbf{Optional:}\newline
    tudresden.ocl20.pivot.ocl2java.test\newline
    tudresden.ocl20.pivot.ocl2java.test.aspectj\newline
    tudresden.ocl20.pivot.tools.template.test\\
    \hline

    \textbf{Ocl22Java Examples} &
    \textbf{Optional:}\newline
    tudresden.ocl20.pivot.examples.royalandloyal.ocl22javacode\newline
    tudresden.ocl20.pivot.examples.simple.ocl22javacode\newline
    tudresden.ocl20.pivot.ocl22java.test.aspectj\\
    \hline

    \textbf{OCL2 Parser} &
    \textbf{Required:}\newline
    tudresden.ocl20.pivot.ocl2parser\newline
    tudresden.ocl20.pivot.parser\newline
    tudresden.ocl20.pivot.parser.ui\newline\newline
    \textbf{Optional:}\newline
    tudresden.ocl20.pivot.ocl2parser.test\\
    \hline
\end{tabular}
\caption{The plug-ins of DresdenOCL (v. 2.2.0) related to their features (part 2).}
\end{table}


\begin{table}[h]
\begin{tabular}{|p{3.5cm}|p{10.5cm}|}
  \hline

  \textbf{Living Example} & \\
  Plug-in Package & tudresden.ocl20.pivot.examples.living\\
  Meta-Model & Java Meta-Model\\
  Model & bin/tudresden/ocl20/pivot/examples/living/ModelProviderClass.class \\
  OCL Expressions & constraints/*.ocl \\
  Model Instance Type & Java \\
  Model Instance & bin/tudresden/ocl20/pivot/examples/living/ModelInstanceProvider\-Class.class \\
  \hline

  \textbf{Simple Example} & \\
  Plug-in Package & tudresden.ocl20.pivot.examples.simple\\
  Meta-Models & Java or MDT UML2\\
  Model & src/tudresden.ocl20.pivot.examples.simple.ModelProviderClass.java, model/simple.uml\\
  OCL Expressions & constraints/*.ocl\\
  Model Instance Type & Java\\
  Model Instance & src/tudresden.ocl20.pivot.examples.simple.instance.Model\-Instance\-ProviderClass.java\\
  Generated\newline AspectJ Code & tudresden.ocl20.pivot.examples.simple.ocl22javacode\\
  \hline

  \textbf{Pain Example} & \\
  Plug-in Package & tudresden.ocl20.pivot.examples.pain\\
  Meta-Model & XSD (XML Schema)\\
  Model & model/pain.008.001.01corrected.xsd\\
  OCL Expressions & constraints/*.ocl\\
  Model Instance Type & XML\\
  Model Instances & modelinstances/BusEx1.xml, \newline modelinstances/BusEx2.xml, \newline modelinstances/Pain8Invalid.xml\\
  \hline

  \textbf{PML Example} & \\
  Plug-in Package & tudresden.ocl20.pivot.examples.pml\\
  Meta-Model & EMF Ecore\\
  Model & model/pml.ecore\\
  OCL Expressions & constraints/*.ocl\\
  Model Instance Type & EMF Ecore\\
  Model Instances & modelinstance/goodModelinstance.pml, \newline modelinstance/badModelinstance.pml\\
  \hline

  \textbf{Royal and Loyal Example} & \\
  Plug-in Package & tudresden.ocl20.pivot.examples.royalandloyal\\
  Meta-Model & MDT UML2\\
  Model & model/royalsandloyals.ecore \newline model/royalsandloyals.uml\\
  OCL Expressions & constraints/*.ocl\\
  Model Instance Type & Java\\
  Model Instance & src/tudresden.ocl20.pivot.examples.royalandloyal.instance.Model\-Instance\-ProviderClass.java\\
  Generated\newline AspectJ Code & tudresden.ocl20.pivot.examples.royalandloyal.ocl22javacode\\
  \hline
\end{tabular}
\caption{The examples provided with DresdenOCL.}
\label{tab:examples}
\end{table}




% Show list of figures. %
\cleardoublepage
\phantomsection
\addcontentsline{toc}{chapter}{List of Figures}
\listoffigures


% Show list of figures. %
\cleardoublepage
\phantomsection
\addcontentsline{toc}{chapter}{List of Tables}
\listoftables


% Show list of listings. %
\cleardoublepage
\phantomsection
\addcontentsline{toc}{chapter}{List of Listings}
\lstlistoflistings{}


% Show list of abbreviations. %
\cleardoublepage
\phantomsection
\addcontentsline{toc}{chapter}{List of Abbreviations}
\chapter*{LIST OF ABBREVIATIONS}

\begin{acronym}[DOT4Eclipse] % Should contain the longest acronym.
\acro{AJDT}{AspectJ Development Tools}
\acro{API}{Application Programming Interface}
\acro{AOP}{Aspect-Oriented Programming}
\acro{DOT}{Dresden OCL Toolkit}
\acro{DOT4Eclipse}{Dresden OCL2 for E\-clip\-se}
\acro{DSL}{Domain-Specific Language}
\acro{Eclipse MDT}{Eclipse Modeling Development Tools}
\acro{EMF}{Eclipse Modeling Framework}
\acro{GMF}{Graphical Modeling Framework}
\acro{GUI}{Graphical User Interface}
\acro{JAR}{Java Archive}
\acro{JDK}{Java Development Kit}
\acro{JRE}{Java Run-time Environment}
\acro{MDT}{Modeling Development Tools}
\acro{MOF}{Meta Object Facility}
\acro{OCL}{Object Constraint Language}
\acro{OMG}{Object Management Group}
\acro{OSGi}{Open Services Gateway initiative}
\acro{SVN}{Subversion}
\acro{UML}{Unified Modeling Language}
\acro{URL}{Uniform Resource Locator}
\acro{XSD}{\acs{XML} Schema Definition}
\acro{XMI}{\acs{XML} Metadata Interchange}
\acro{XML}{Extensible Markup Language}
\end{acronym}


% Show Bibliography. %
\begin{flushleft}
\cleardoublepage
\phantomsection
\addcontentsline{toc}{chapter}{References}
\bibliographystyle{alphadin}
\bibliography{manual}
\end{flushleft}

