\begin{abstract}
\addcontentsline{toc}{chapter}{Abstract}
This document contains the documentation of DresdenOCL. In the first part, the general use of DresdenOCL is explained. Afterwards, use cases like \acs{OCL} interpretation and code generation are presented. The second part contains the technical documentation of DresdenOCL, like its architecture and the adaptation of further meta-models to DresdenOCL.

Please be aware that DresdenOCL is a project developed at the Technische Universit�t Dresden, Software Technology Group. Parts of the project have been designed and implemented during student theses and have been developed as prototypes only. Thus, DresdenOCL is far from being complete. To report bugs and errors or request additional features or answers to specific questions visit DresdenOCL's website \cite{WWW:toolkit} or the project site at \keyword{Sourceforge} \cite{WWW:toolkitSourceforge}.

The procedure's described in this manual were run and tested with \keyword{Eclipse 3.5} \cite{WWW:eclipse}. We recommend to use the \keyword{Eclipse Modeling Tools Edition} which contains all required plug-ins to run DresdenOCL.\footnote{Apart from the \acs{AJDT} that are required to run some of the provided examples.} Otherwise you need to install at least the plug-ins enlisted in Table \ref{tab:software}. Alternatively, DresdenOCL may be used as a stand-alone library for Java. If you want to use the stand-alone distribution, you cannot use the \acs{GUI}s and editors provided with DresdenOCL since the GUI elements depend on Eclipse. The use of the stand-alone distribution is documented in Chapter~\ref{chapter:standalone}.

\end{abstract}