\documentclass[12pt,a4paper,oneside]{report}
\usepackage{graphicx}
\usepackage{isolatin1}
\usepackage{rotating}


\oddsidemargin=0cm
\evensidemargin=0cm
\marginparwidth=0cm
\textwidth=15cm

\begin{document}
\graphicspath {{/}}
\pagestyle{empty}
\begin{titlepage}
\begin{center}
\verb$ $\\
\addvspace{2cm}
\Large{Diplomarbeit zum Thema}\\
\bigskip
\Huge Werkzeugunterst�tzung f�r die �berpr�fung der
Einhaltung von OCL-Gesch�ftsregeln in Java-Programmen\\
\vspace{3cm}
\Large
bearbeitet von Ralf Wiebicke\\
geboren am 19. Februar 1975 in Dresden\\
\bigskip
an der\\
\bigskip
Technischen Universit�t Dresden\\
Fakult�t Informatik\\
Lehrstuhl Softwaretechnologie\\
\end{center}
\vspace{3cm}
\Large
Betreuerin: Dr. B. Demuth\\
Verantw. Hochschullehrer: Prof. Dr. H. Hu�mann\\
Eingereicht am 15.12.2000
\end{titlepage}

\verb$ $

\begin{center}
\verb$ $\\
\vspace{4cm}
\Huge Utility Support for\\Checking OCL Business Rules in \\Java Programs\\
\vspace{3cm}
\Large
by Ralf Wiebicke\\
\bigskip
Dresden University of Technology\\
Department of Computer Science\\
Software Engineering Group\\
\end{center}
\vspace{3cm}
\normalsize

\newpage

\verb$ $

\pagestyle{empty}

\verb$ $
\vspace{1cm}

Ich erkl�re, da� ich die vorliegende Arbeit selbst�ndig und nur unter Verwendung
der angegebenen Literatur und Hilfsmittel angefertigt habe.

\vspace{2cm}
Dresden, den 15.12.2000


\end{document}
